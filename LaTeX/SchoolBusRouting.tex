\documentclass[times, utf8, seminar, numeric]{fer}
\usepackage{booktabs}
 \usepackage{url}

\begin{document}

% Ukljuci literaturu u seminar
\nocite{*}

% TODO: Navedite naslov rada.
\title{Usmjeravanje školskih autobusa s odabirom autobusnih stanica}

% TODO: Navedite vaše ime i prezime.
\author{Zvonimir Jurelinac, Andrija Miličević}

% TODO: Navedite ime i prezime voditelja.
\voditelj{Izv. Prof. Dr. Sc. Lea Skorin-Kapov}

\maketitle

\tableofcontents

\chapter{Opis problema}
Problem usmjeravanja školskih autobusa je varijacija problema usmjeravanja vozila, koji se od osnovnog problema razlikuje po tome što nije striktno zadano koja je sve mjesta potrebno posjetiti, već je dana lista potencijalnih stanica, i na algoritmu je da sam odabere koje će od njih i kojim redom posjetiti, te da pritom zadovolji sva zadana ograničenja.

Konkretno, u problemu su definirane lokacije potencijalnih stanica školskih autobusa, lokacije svih učenika koji školskim autobusima putuju do škole, te lokacija same škole u koju svi studenti moraju u što kraćem vremenu prispjeti. Zadatak se sastoji u tome da se odabere skup stanica koje će školski autobusi posjetiti, potom odredi koji će učenici propješaćiti i biti pokupljeni na kojim stanicama, te na kraju, koji će autobusi i kojim redom posjetiti sve potrebne stanice. Svi učenici imaju (jednak) ograničen radijus kretanja, tako da im se može dodijeliti samo neka od stanica u njihovu dometu. Također, svi autobusi imaju ograničen broj mjesta, tako da ne mogu odjednom prevesti više učenika od onoga koliki im je kapacitet. Isto tako, jednu stanicu može posjetiti samo jedan autobus, i svi učenici koji čekaju na njoj moraju se tada ukrcati u taj autobus.

\chapter{Opis primjenjenog algoritma}
Tokom rada na projektu isprobano je mnoštvo različitih pristupa i algoritama. Pristup prikazan u nastavku dao je najbolje rezultate.

Algoritam djeluje neovisno na nekoliko razina. Prva razina je pridruživanje svakom studentu autobusnu stanicu pazeći pritom da se ne premaši granica maksimalnog kapaciteta jednog autobusa. Na drugoj razini iz skupa stanica koje se posjećuju odabire se redosljed njihovog posjećivanja. Na trećoj razini izvodi se lokalna pretraga koja pokušava poboljšati trenutni redosljed. Konačno na četvrtoj razini iz prethodnog redosljeda na optimalan način se raspoređuju autobusi koristeći dinamičko programiranje.

Algoritam izbacuje stanice koje su napunile maksimalni kapacitet te ih na kraju samo dodaje u rješenje. Ukoliko postoji 11 ili manje stanica koje nisu napunile maksimalni kapacitet, umjesto lokalne pretrage se vrši pretraga svih mogućih permutacija obilaska stanica.

\section{Prva razina - pridruživanje autobusnih stanica studentima}
U praksi nam se pokazalo kako je dobro započeti rješenje pohlepnim pristupom te potom nekom usmjerenom pretragom poboljšati to rješenje. Tako se i u prvom koraku ovdje predloženog algoritma pokušava pridjeliti autobusne stanice studentima.

Dva kriterija su bila glavna za pridruživanje. Prvi kriterij je da se ukupno odabere što manje autobusnih stanica. Drugi kriterij je da su te stanice što bliže školi. Naime očito je isplativije da studenti hodaju što bliže školi nego da autobus ide duljim putem po njih. A i intuicija govori da ukoliko se posjećuje manje autobusnih stanica, vjerojatnije je da će ukupan put biti manji.

Algoritam gradi početno rješenje tako da kreće od stanica najbližih školi te pokušava naći stanicu koju mogu posjetiti najveći broj studenata. Ukoliko postoji više studenata koji mogu posjetiti određenu stanicu nego što je maksimalni kapacitet, algoritam na slučajan način odabire maksimalni podskup studenata koji će dodjeliti toj stanici. Ovo omogućuje da će se rješenja međusobno razlikovati.

Konačni algoritam je bio vrlo brz što nam je omogućilo da možemo mnogo puta pokretati algoritam s različitim početnim rješenjima. Zato je u pohlepno građenje početnog rješenje uveden šum koji neće uvijek uzeti najbolje rješenje prema ovoj pohlenoj heuristici. Postoji vjerojatnost da će se neke najbliže stanice ponekad preskočiti.

Ostali pokušaji generiranja početnog rješenja na drukčiji način nisu usporedivi s prethodno navedenim pristupom.

\chapter{Pseudokod primjenjenog algoritma}
Pseudokod

\chapter{Rezultati}

\chapter{Zaključak}

%\bibliography{literatura}
%\bibliographystyle{fer}

\chapter{Sažetak}
Sažetak.

\end{document}
