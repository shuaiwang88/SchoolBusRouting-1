\documentclass[times, utf8, seminar, numeric]{fer}
\usepackage{booktabs}
 \usepackage{url}

\begin{document}

% Ukljuci literaturu u seminar
\nocite{*}

% TODO: Navedite naslov rada.
\title{Usmjeravanje školskih autobusa s odabirom autobusnih stanica}

% TODO: Navedite vaše ime i prezime.
\author{Zvonimir Jurelinac, Andrija Miličević}

% TODO: Navedite ime i prezime voditelja.
\voditelj{Izv. Prof. Dr. Sc. Lea Skorin-Kapov}

\maketitle

\tableofcontents

\chapter{Opis problema}
Problem usmjeravanja školskih autobusa je varijacija problema usmjeravanja vozila, koji se od osnovnog problema razlikuje po tome što nije striktno zadano koja je sve mjesta potrebno posjetiti, već je dana lista potencijalnih stanica, i na algoritmu je da sam odabere koje će od njih i kojim redom posjetiti, te da pritom zadovolji sva zadana ograničenja.

Konkretno, u problemu su definirane lokacije potencijalnih stanica školskih autobusa, lokacije svih učenika koji školskim autobusima putuju do škole, te lokacija same škole u koju svi studenti moraju u što kraćem vremenu prispjeti. Zadatak se sastoji u tome da se odabere skup stanica koje će školski autobusi posjetiti, potom odredi koji će učenici propješaćiti i biti pokupljeni na kojim stanicama, te na kraju, koji će autobusi i kojim redom posjetiti sve potrebne stanice. Svi učenici imaju (jednak) ograničen radijus kretanja, tako da im se može dodijeliti samo neka od stanica u njihovu dometu. Također, svi autobusi imaju ograničen broj mjesta, tako da ne mogu odjednom prevesti više učenika od onoga koliki im je kapacitet. Isto tako, jednu stanicu može posjetiti samo jedan autobus, i svi učenici koji čekaju na njoj moraju se tada ukrcati u taj autobus.

\chapter{Opis primjenjenog algoritma}
Rad.

\chapter{Pseudokod primjenjenog algoritma}
Pseudokod

\chapter{Rezultati}

\chapter{Zaključak}

%\bibliography{literatura}
%\bibliographystyle{fer}

\chapter{Sažetak}
Sažetak.

\end{document}
